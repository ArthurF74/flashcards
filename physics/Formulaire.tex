\documentclass[10pt,a4paper]{article}
\usepackage[utf8]{inputenc}
\usepackage[french]{babel}
\usepackage[T1]{fontenc}
\usepackage{amsmath}
\usepackage{amsfonts}
\usepackage{amssymb}
\usepackage{svg}
\usepackage{multicol}
\usepackage{textcomp}

\title{Formulaire de physique}

\begin{document}
\section{Physique du point Matériel}

\subsection{Noms de variables}

\begin{tabular}{|c|c|c|}
\hline 
Grandeur & Qu'est-cequec'estquoi & Unités \\ 
\hline 
$M$ & Masse & $[kg]$ \\
\hline
$x(t)$ & Position en fonction du temps & $[m]$ \\ 
\hline
$v(t) = \dot{x}(t)$ & Vitesse & $[\frac{m}{s}]$ \\
\hline
$a(t) = \ddot{x}(t)$ & Acceleration Angulaire & $[\frac{rad}{s^2}]$ \\
\hline
$F_{cause}$ & Force associée à une cause & $[N] = [\frac{kg*m}{s^2}]$ \\
\hline
$p(t) = mv(t)$ & Quantité de mouvement & $[\frac{kg*m}{s}]$ \\
\hline
$E_{cin}$ & Energie cinétique du point & $[J] = [\frac{kg*m^2}{s^2}$ \\
\hline
\end{tabular}

\subsection{Formules}

$$\sum F_{ext} = ma$$

\subsection{Systemes de coordonnées}
\subsubsection{Coordonnées Sphériques}

\begin{multicols}{3}

\begin{align*}
x & = \rho \sin \phi \cos \theta \\
y & = \rho \sin \phi \sin \theta \\
z & = \rho \cos \phi
\end{align*}

\columnbreak

\includesvg[width = 0.3\textwidth]{sphere}

\columnbreak


\begin{align*}
\rho   &= \sqrt{x^2+y^2+z^2}\\
\phi &= \arccos(z/\rho)\\
\theta &= \begin{cases}\arccos\frac{x}{\sqrt{x^2+y^2}} & \mathrm{si}\ y\geq{0} \\ 2\pi-\arccos\frac x{\sqrt{x^2+y^2}} & \mathrm{si}\ y < 0\end{cases}\\
\theta &= \arctan(y/x)
\end{align*}

\end{multicols}

Vecteur rayon et dérivées:

\begin{align*}
\vec{r} &= \rho \hat{e}_{\rho} \\
\dot{\vec{r}} &= \dot{\rho}\hat{e}_{\theta} + \rho\dot{\phi}\sin\theta\hat{e}_{\phi} = \vec{v} \\
\begin{split}
\ddot{\vec{r}} &= \left( \ddot{\rho} - \rho\dot\theta^2 - \rho\dot\phi^2\sin^2\theta \right)\mathbf{\hat{e}_\rho} \\
&+ \left( \rho\ddot\theta + 2\dot{\rho}\,\dot\theta - \rho\dot\phi^2\sin\theta\cos\theta \right) \mathbf{\hat{e}_\theta}\\
&+ \left( \rho\ddot\phi\,\sin\theta + 2\dot{\rho}\,\dot\phi\,\sin\theta + 2 \rho\dot\theta\,\dot\phi\,\cos\theta \right) \mathbf{\hat{e}_\phi} 
\end{split}
\end{align*}

\subsubsection{Coordonnées Cylindriques}

\begin{multicols}{3}

\begin{align*}
x &= r  \cos\theta \\
y &= r  \sin\theta \\
z &= z
\end{align*}

\columnbreak

\includesvg[width = 0.3\textwidth]{cylindre}

\columnbreak

\begin{align*}
r & = \sqrt{x^2 + y^2}\\
\theta & = \arctan\left({y \over x}\right) \\
z & = z
\end{align*}

\end{multicols}

\section{Physique du corps Solide}

\subsection{Noms de variables}

\begin{tabular}{|c|c|c|}
\hline 
Grandeur & Qu'est-cequec'estquoi & Unités \\ 
\hline 
$I$ & Repartition de la masse pondérée par la distance au carré & $[kg*m^2]$ \\ 
\hline
$\omega$ & Vitesse angulaire , vecteur parralèle à l'axe de rotation. & $[\frac{rad}{s}]$ \\ 
\hline
$L$ & Moment cinétique & $[\frac{kg*m^2}{s}]$ \\
\hline
$\alpha$ & Acceleration Angulaire & $[\frac{rad}{s^2}]$ \\
\hline
\end{tabular}

\subsection{Formules}
$\sum \vec{M_{ext}} = I\vec{\alpha}$ ,
$\frac{d\vec{L}}{dt} = \sum \vec{M_{ext}}$,
$\vec{v} = \vec{\omega} \times \vec{r}$,
$\vec{a_t} = \vec{\alpha} \times \vec{r}$,
$I = \sum \limits_M d^2 dm$,



\end{document}