% -*- coding-system:utf-8 
% LATEX PREAMBLE --- needs to be imported manually
\documentclass[12pt]{article}
\special{papersize=3in,5in}
\usepackage[utf8]{inputenc}
\usepackage{amssymb,amsmath}
\pagestyle{empty}
\setlength{\parindent}{0in}

%%% commands that do not need to imported into Anki:
\usepackage{mdframed}
\newcommand*{\xfield}[1]{\begin{mdframed}\centering #1\end{mdframed}\bigskip}
\newenvironment{note}{}{}
% END OF THE PREAMBLE
\begin{document}

\begin{note}
    \xfield{What is logically equivalent to $p \rightarrow q$}
    \xfield{$\neg p \vee q$}
\end{note}

\begin{note}
    \xfield{What's De Morgan's law}
    \xfield{
        $\neg (p\wedge q) \equiv \neg p \vee \neg q$  \\
        $\neg (p \vee q) \equiv \neg p \wedge \neg q$
    }
\end{note}

\begin{note}
    \xfield{What is a predicate ?}
    \xfield{A "formula that involves some variables, variables which come from a domain.}
\end{note}

\begin{note}
    \xfield{What are the two quantifiers ?}
    \xfield{$\forall$ which means "for all" \& $\exists$ which means "there exists"}
\end{note}

\begin{note}
    \xfield{Give the rules of negation of quantifiers.}
    \xfield{
        $\neg (\forall x P(x)) \equiv \exists x \neg P(x)$ \\
        $\neg (\exists x P(x) \equiv \forall x \neg P(x)$
    }
\end{note}

\begin{note}
    \xfield{What does modus ponens implies ? (Rule of inference in propositional logic) \begin{tabular}{c@{\,}l@{}}                          & $p$ \\& $p \to q$ \\\cline{2-2}    $\therefore$         & \ ??? \\  \end{tabular}}
    \xfield{
        \begin{tabular}{c@{\,}l@{}} 
                                 & $p$ \\
        & $p \to q$ \\\cline{2-2}
            $\therefore$         & $q$ \\  \end{tabular}
    }
\end{note}

\begin{note}
    \xfield{What does modus tollens implies ? (Rule of inference in propositional logic) \begin{tabular}{c@{\,}l@{}}                          & $\neg p$ \\& $p \to q$ \\\cline{2-2}    $\therefore$         & \ ??? \\  \end{tabular}}
    \xfield{
        \begin{tabular}{c@{\,}l@{}} 
                                 & $\neg p$ \\
        & $p \to q$ \\\cline{2-2}
            $\therefore$         & $q$ \\  \end{tabular}
    }
\end{note}

\begin{note}
    \xfield{What does simplification implies ? (Rule of inference in propositional logic) \begin{tabular}{c@{\,}l@{}} & $p \wedge q$ \\\cline{2-2}    $\therefore$         & \ ??? \\  \end{tabular}}
    \xfield{
        \begin{tabular}{c@{\,}l@{}}         
        & $p \wedge q$ \\\cline{2-2}
            $\therefore$         & \ p \\  \end{tabular}
    }
\end{note}

\begin{note}
    \xfield{What does addition rule implies ? (Rule of inference in propositional logic) \begin{tabular}{c@{\,}l@{}} & $p$ \\\cline{2-2}    $\therefore$         & \ ??? \\  \end{tabular}}
    \xfield{
        \begin{tabular}{c@{\,}l@{}} 
        & $p$ \\\cline{2-2}
            $\therefore$         & \ $p \vee q$ \\  \end{tabular}
    }
\end{note}

\begin{note}
    \xfield{
        What does conjonction implies ? (Rule of inference in propositional logic) \begin{tabular}{c@{\,}l@{}}
        & $p$ \\
        & $q$ \\\cline{2-2}    $\therefore$         & \ ??? \\  \end{tabular}
    }
    \xfield{
        \begin{tabular}{c@{\,}l@{}}& $p$ \\
        & $q$ \\\cline{2-2}    $\therefore$         & \ $p \wedge q$ \\  \end{tabular}
    }
\end{note}

\begin{note}
    \xfield{
        What does the hypotetical syllogism implies ? (Rule of inference in propositional logic) \begin{tabular}{c@{\,}l@{}}
        & $p \to q$ \\
        & $q \to r$ \\\cline{2-2}    $\therefore$         & \ ??? \\  \end{tabular}
    }
    \xfield{
        \begin{tabular}{c@{\,}l@{}}& $p \to q$ \\
        & $q \to r$ \\\cline{2-2}    $\therefore$         & \ $p \to r$ \\  \end{tabular}
    }
\end{note}

\begin{note}
    \xfield{
        What does the disjunctive syllogism implies ? (Rule of inference in propositional logic) \begin{tabular}{c@{\,}l@{}}
        & $p \vee q$ \\
        & $\neg p$ \\\cline{2-2}    $\therefore$         & \ ??? \\  \end{tabular}
    }
    \xfield{
        \begin{tabular}{c@{\,}l@{}}& $p \vee q$ \\
        & $\neg p$ \\\cline{2-2}    $\therefore$         & \ $q$ \\  \end{tabular}
    }
\end{note}

\begin{note}
    \xfield{
        What does the resolution implies ? (Rule of inference in propositional logic) \begin{tabular}{c@{\,}l@{}}
        & $p \vee q$ \\
        & $\neg p \vee r$ \\
        \cline{2-2}    $\therefore$  & ???
        \end{tabular}
        }
    \xfield{
        \begin{tabular}{c@{\,}l@{}}
        & $p \vee q$ \\
        & $\neg p \vee r$ \\
        \cline{2-2}    $\therefore$         & \ $q \vee r$ \\  \end{tabular}
    }
\end{note}

\begin{note}
    \xfield{What does $p \to q$ implies ?}
    \xfield{$\neg p \to \neg q$}
\end{note}

\begin{note}
    \xfield{
        What does the universal generalization implies ? (Rule of inference in propositional logic) \begin{tabular}{c@{\,}l@{}}
        & $P(x)$ for all $x \in D$ \\\cline{2-2}    $\therefore$         & \ ??? \\  \end{tabular}
    }
    \xfield{ \begin{tabular}{c@{\,}l@{}}& $P(x)$ for all $x \in D$ \\\cline{2-2}    $\therefore$         & \ $\forall x P(x)$ \\  \end{tabular}}
\end{note}

\begin{note}
    \xfield{
        What does the universal specialization implies ? (Rule of inference in propositional logic) \begin{tabular}{c@{\,}l@{}}
        & $\forall x$ \\\cline{2-2}    $\therefore$         & \ ??? \\  \end{tabular}
    }
    \xfield{ \begin{tabular}{c@{\,}l@{}}& $\forall x$ \\\cline{2-2}    $\therefore$         & \ $P(d)$ for any $d \in D$ \\  \end{tabular}}
\end{note}

\begin{note}
    \xfield{
        What does the existential generalization implies ? (Rule of inference in propositional logic) \begin{tabular}{c@{\,}l@{}}
        & $P(d)$ for some $d \in D$ \\\cline{2-2}    $\therefore$         & \ ??? \\  \end{tabular}
    }
    \xfield{ \begin{tabular}{c@{\,}l@{}}& $P(d)$ for some $d \in D$ \\\cline{2-2}    $\therefore$         & \ $\exists x P(x)$ \\  \end{tabular}}
\end{note}

\begin{note}
    \xfield{
        What does the existential simplification implies ? (Rule of inference in propositional logic) \begin{tabular}{c@{\,}l@{}}
        & $\exists x P(x)$ \\\cline{2-2}    $\therefore$         & \ ??? \\  \end{tabular}
    }
    \xfield{ \begin{tabular}{c@{\,}l@{}}& $\exists x P(x)$ \\\cline{2-2}    $\therefore$         & \  $P(d)$ for some $d \in D$ \\  \end{tabular}}
\end{note}

\begin{note}
    \xfield{What is the $\mathcal P \left({A}\right)$ (powerset of A)}
    \xfield{It is the set that consist of all subset of A eg :Let $A = \{1,2,3\}$, then $\mathcal P \left({A}\right)\ =\ (\varnothing,\{1,2,3\},\{1\},\{2\},\{3\},\{1,2\},\{1,3\},\{2,3\}\}$}
\end{note}

\begin{note}
    \xfield{
        Define the cartesian product of two sets.
        Define its cardinality.
    }
    \xfield{
        Let $A$ and $B$ two sets,\\
        $A \times B\ =\ \{(a,b) : a \in A \wedge b \in B\}$\\
        Eg : $A =\{1,2\} B = \{3,4\}$\\
        $A \times B = \{(1,3),(1,4),(2,3),(2,4)\}$\\
        $\left\vert A \times B\right\vert = \left\vert A \right\vert \cdot \left\vert B \right\vert $
    }
\end{note}

\begin{note}
    \xfield{What is the condition for two sets to have the same cardinality ?}
    \xfield{For $\left\vert A \right\vert = \left\vert B \right\vert $ to be true, there must exist a bijection $f\ :\ A \to B$ }
\end{note}

\begin{note}
    \xfield{What is $\left\vert \mathbb{N} \right\vert$}
    \xfield{ $\left\vert \mathbb{N} \right\vert =  \aleph_0$}
\end{note}

\begin{note}
    \xfield{What are the conditions for a set to be countable ?}
    \xfield{A set is countable if it is finite or if there exist a bijection with $\mathbb{N}$}
\end{note}

\begin{note}
    \xfield{$\sum_{i=0}^{k} i = ?$}
    \xfield{$\sum_{i=0}^{k} i = \frac{k(k+1))}{2}$}
\end{note}

\begin{note}
    \xfield{$\sum_{i=0}^{k} r^i = ?$}
    \xfield{$\sum_{i=0}^{k} r^i = \frac{r^{k+1} - 1}{r - 1}$}
\end{note}

\begin{note}
    \xfield{What does $x \to \llcorner x \urcorner$ mean ?}
    \xfield{round to closest integer (if $\frac{1}{2}$, then round down)}
\end{note}

\begin{note}
    \xfield{What does $x \to \lfloor x \rfloor$ mean ?}
    \xfield{map to largest integer $\le x$}
\end{note}

\begin{note}
    \xfield{What does $x \to \lceil x \rceil$ mean ?}
    \xfield{map to smallest integer $\ge x$}
\end{note}

\begin{note}
    \xfield{What does $x \to [ x]$ mean ?}
    \xfield{map to integer part}
\end{note}

\begin{note}
    \xfield{What does $f(x) = \mathcal{O}(g(x))$ means ?}
    \xfield{
        We say that  $f(x) = \mathcal{O}(g(x))$ if \\
        $\exists k,c \in \mathbb{R}$ s.t. $\forall x \ge k\ \left\vert f(x)\right\vert \le c \cdot \left\vert g(x)\right\vert $ \\
        $\left\vert g(x)\right\vert$ is up to a constant factor \underline{eventually} an upper bound of $\left\vert f(x)\right\vert $
    }
\end{note}

\begin{note}
    \xfield{What does $f(x) = \Omega(g(x))$ means ?}
    \xfield{
        We say that  $f(x) = \Omega(g(x))$ if \\
        $\exists k,c > 0 \in \mathbb{R}$ s.t. $\forall x \ge k\ \left\vert f(x)\right\vert \ge c \cdot \left\vert g(x)\right\vert $ \\
        $\left\vert g(x)\right\vert$ is up to a constant factor \underline{eventually} a lower bound of $\left\vert f(x)\right\vert $
    }
\end{note}

\begin{note}
    \xfield{What does $f(x) = \Theta(g(x))$ means ?}
    \xfield{We say that $f(x) = \Theta(g(x))$ if $\{f(x)\ is\ \mathcal{O}(g(x))\} \wedge \{f(x)\ is\ \Omega (g(x))\}$We say that $f(x) = \Theta(g(x))$ if $\{f(x)\ is\ \mathcal{O}(g(x))\} \wedge \{f(x)\ is\ \Omega (g(x))\}$}
\end{note}

\begin{note}
    \xfield{What does $f(x) = o(g(x))$ means ?}
    \xfield{We say that $f(x) = o(g(x))$ if $\lim\limits_{x \to \infty} \frac{\left\vert f(x) \right\vert}{\left\vert g(x) \right\vert} = 0$}
\end{note}

\begin{note}
	\xfield{Define the pigeonhole principle}
	\xfield{If $n$ objects are distributed over $m$ places, and if $n > m$, then some place receives at least two objects}
\end{note}

\begin{note}
	\xfield{What is Fermat's little theorem}
	\xfield{Let $p$ be prime, then $\forall a \in \mathbb{Z},\ a^p \equiv a\ mod\ p$}
\end{note}

\begin{note}
	\xfield{}
	\xfield{}
\end{note}

\begin{note}
	\xfield{Define the mathematical induction and the strong induction :}
	\xfield{\begin{itemize}
		\item $(P(0) \wedge \forall n \in \mathbb{N}_{\ge 0} P(n) \rightarrow P(n+1)) \rightarrow \forall n \in \mathbb{N}_{\ge 0} P(n)$
		\item $(P(0) \wedge \forall n \in \mathbb{N}_{\ge 0} P(1) \wedge P(2)\wedge ... \wedge P(n) \rightarrow P(n+1)) \rightarrow \forall n \in \mathbb{N}_{\ge 0} P(n)$
	\end{itemize} }
\end{note}

\begin{note}
Counting :
	\xfield{\begin{enumerate}
		\item Define the product rule
		\item Define the sum rule
	\end{enumerate} }
	\xfield{\begin{enumerate}
		\item Ff a task can be accomplished by first doing task $A$ and then task $B$, what $A$ does can be done in $a$ ways and $B$ in $b$ ways and there is no depedence between $A$ and $B$, then the overall task can be done in $a\cdot b$ ways. 
		\item If a task can be accomplished by either doing $A$ \underline{or} (exclusive) $B$, then the overall task can be done in $a+b$ steps.
	\end{enumerate} }
\end{note}

\begin{note}
	How may ways are there to pick objects from a set of size n ?
	\xfield{\begin{enumerate}
		\item If the order matters and the repetitions aren't allowed
		\item If the order matters and the repetitions are allowed
		\item If the order doesn't matter and the repetitions aren't allowed
		\item If the order doesn't matter and the repetitions are allowed
	\end{enumerate} }
	\xfield{\begin{enumerate}
		\item Permutation without repetition :\\
			$P(n,r) = \frac{n\cdot(n-1)\cdot(n-2)\cdot ...\cdot(n-r+1)\cdot(n-r)\cdot ...\cdot 2\cdot 1}{(n-r+1)\cdot (n-r)\cdot ... \cdot 2 \cdot 1} = \frac{n!}{(n-r)!}$
		\item Combination without repetition :\\
		$C(n,r) = \binom{n}{r}\ (=\frac{n!}{(n-r)!r!})$
		\item Permutation with repetition :\\
		$n\cdot n\cdot n \cdot n \cdot ... \cdot n = n^r$
		\item Combination with repetition :\\
		$\binom{r+n-1}{n-1} = \frac{(r+n-1)!}{(r+(n-1)-(n-1))!(n-1)!} = \frac{(r+n-1)!}{r!(n-1)!} = \binom{r+n-1}{r}$
	\end{enumerate} }
\end{note}

\begin{note}
	\xfield{\begin{itemize}
	\item Give Pascal's identity for $\binom{n}{k}$
	\item Give Vandermonde's identity for $\binom{m+n}{r}$
	\item $\binom{n}{r}\binom{r}{k}$ ?
	\end{itemize} }
	\xfield{\begin{itemize}
	\item $\binom{n}{k} = \binom{n-1}{k} + \binom{n-1}{k-1}$
	\item  $\binom{m+n}{r} = \sum\limits^{r}_{k=0} \binom{n}{r-k}\binom{m}{k}$
	\item $\binom{n}{r}\binom{r}{k}=\binom{n}{k}\binom{n-k}{r-k}$
	\end{itemize} }
\end{note}

\begin{note}
	\xfield{}
	\xfield{ }
\end{note}


\end{document}