% -*- coding-system:utf-8 
% LATEX PREAMBLE --- needs to be imported manually
\documentclass[12pt]{article}
\special{papersize=3in,5in}
\usepackage[utf8]{inputenc}
\usepackage{amssymb,amsmath}
\pagestyle{empty}
\setlength{\parindent}{0in}

%%% commands that do not need to imported into Anki:
\usepackage{mdframed}
\newcommand*{\xfield}[1]{\begin{mdframed}\centering #1\end{mdframed}\bigskip}
\newenvironment{note}{}{}
% END OF THE PREAMBLE
\begin{document}
\begin{note}
	\xfield{Calculer $sin(\frac{\pi}{4})$ et $cos(\frac{\pi}{4})$}
	\xfield{On peut calculer facilement ces valeurs en prenant un triangle isocèle de côtés isocèles 1, il aura deux angles de $\frac{\pi}{4}$, un angle droit et une hypothénuse de $\sqrt{2}$ alors : \\
	$\sin(\frac{\pi}{4}) = \frac{1}{\sqrt{2}} = \frac{\sqrt{2}}{2}$\\
	$\cos(\frac{\pi}{4}) = \frac{\sqrt{2}}{2}$}
\end{note}
\begin{note}
	\xfield{Calculer
$sin(\frac{\pi}{6})$,  $cos(\frac{\pi}{6})$ , $sin(\frac{\pi}{3})$ et $cos(\frac{\pi}{3})$}
	\xfield{On peut calculer facilement ces valeurs en prenant un triangle équilatéral de côté 1, il aura alors des angles de $\frac{\pi}{3}$ et si l'on coupe le triangle en deux, cela formera un triangle rectangle avec un angle de $\frac{\pi}{6}$ qui aura :
	\begin{itemize}
	\item une hypothénuse de 1
	\item  un côté adjacent à $\frac{\pi}{3}$ et opposé à $\frac{\pi}{6}$ de $\frac{1}{2}$
	\item un côté opposé à $\frac{\pi}{3}$ et adjacent à $\frac{\pi}{6}$ de $\sqrt{(1^2-(\frac{1}{2})^2} = \frac{\sqrt{3}}{2}$
	\end{itemize}
	On a donc :\\
	$sin(\frac{\pi}{6}) = \frac{1}{2}$, $cos(\frac{\pi}{6}) = \frac{\sqrt{3}}{2}$ \\
	$sin(\frac{\pi}{3}) = \frac{\sqrt{3}}{2}$, $cos(\frac{\pi}{3}) = \frac{1}{2}$}
\end{note}
\begin{note}
	\xfield{Quelles sont les trois règles définissant une classe d'équivalence ?}
	\xfield{Relation d'équivalence sur $X$ notée $\sim$ \\
	R1) $x \sim x\ \ \forall x \in X$ (réflexive)\\
	R2) $x \sim y \rightarrow y\sim x$ (symétrique)\\
	R3) $x \sim y,y \sim z \rightarrow x \sim z$ (transitive)}
\end{note}
\begin{note}
	\xfield{Expliquer ce qu'est la représentation privilégiée d'un nombre rationnel et ce pour quoi elle peut-être utile.}
	\xfield{Le représentantant privilégié d'un nombre rationnel $x \in \mathbb{Q}$ est $\frac{p}{q}$ avec $q > 0$ et pgcd$(\lvert p \lvert,q) = 1$\\
	Soit $x = \frac{a}{1}$, $y = \frac{b}{1}$ alors $x+y = \frac{a+b}{1}$ et $x \cdot y = \frac{a\cdot b}{1}$\\
	Cette représentation peut être utiler pour montrer qu'un nombre n'est pas un rationnel.}
\end{note}
\begin{note}
	\xfield{Définir le raisonnement par récurrence (principe d'induction)}
	\xfield{\begin{enumerate}
	\item si $P(n_0)$ est vrai pour $n_0 \in \mathbb{N}$ (initialisation)
	\item et si pour tout $n \ge n_0$ $P(n) \rightarrow P(n+1)$ (le pas d'induction)
	\end{enumerate}
	alors $P(n)$ est vrai pour tout $n \ge n_0$}
\end{note}
\begin{note}
	\xfield{Définir le produit cartésien}
	\xfield{Soit $X,Y$ des ensembles \\
	$X \times Y = \{ (x,y) : x \in X, y \in Y \} $\\
	Exemple : $X = \{ 1,2 \} , Y = \{ 3,4 \} $\\
	$X \times Y = \{ (1,3),(1,4),(2,3),(2,4)\}$ \\
	attention : $X \times Y \neq Y \times X$ en général.}
\end{note}
\begin{note}
    \xfield{
        Définir ce qu'est un minorant et un majorant pour un ensemble.\\
        Définir ce que cela signifie si un ensemble est minoré ou majoré.\\
        Finalement, définir ce que cela signifie si un ensemble est borné.}
    \xfield{
        \underline{minorant :} $a\ \in \mathbb{R}$ est un minorant de $A\ \subset\ \mathbb{R},\ A\ \neq\ \varnothing$ si $a\ \le\ x,\ \forall x\ \in\ A$.\\        
        \underline{majorant :} $a\ \in \mathbb{R}$ est un majorant de $A\ \subset\ \mathbb{R},\ A\ \neq\ \varnothing$ si $a\ \ge\ x,\ \forall x\ \in\ A$.\\
        \underline{minoré ou borné inférieurement:} $A\ \subset\ \mathbb{R}$ est minoré si $A$ admet un minorant.\\
        \underline{majoré ou borné supérieurement:} $A\ \subset\ \mathbb{R}$ est majoré si $A$ admet un majorant.\\
        \underline{borné :} $A\ \subset\ \mathbb{R},\ A\ \neq\ \varnothing$ est borné si A est majoré et minoré.}
\end{note}

\begin{note}
    \xfield{
        Définir ce qu'est l'infimum d'un ensemble.\\
        Définir ce qu'est le minimum d'un ensemble.\\
    }
    \xfield{
        \underline{Infimum : } un minorant de $a \in\ \mathbb{R}$ est appelé infimum ou borne inférieure de $A\ \subset\ \mathbb{R},\ A\ \neq\ \varnothing$, noté :\\        
        \begin{center}
              $a\ =\ inf(A)$,\\
        \end{center}
        si a est le plus grand minorant de A, c'est-à-dire, si tout minorant b de A satifsait b $\le$ a\\    
        \underline{Minimum : }\\
        $si\ a=inf(A)\ \in A,\ alors\ inf(A)\ =:\ min(A)$ }
\end{note}

\begin{note}
    \xfield{
        Définir ce qu'est le supremum d'un ensemble.
        Définir ce qu'est le maximum d'un ensemble.
    }
    \xfield{
        \underline{supremum : } un majorant de $a \in\ \mathbb{R}$ est appelé supremum ou borne supérieure de $A\ \subset\ \mathbb{R},\ A\ \neq\ \varnothing$, noté : \\
                $a\ =\ sup(A)$,\\
        si a est le plus petit majorant de A, c'est-à-dire, si tout majorant b de A satifsait b $\ge$ a.\\        
        \underline{Maximum : }\\
        $si\ a=sup(A)\ \in A,\ alors\ sup(A)\ =:\ max(A)$.
    }
\end{note}

\begin{note}
    \xfield{Définir ce que signifie si un ensemble est ouvert, et ce qu'est son intérieur.}
    \xfield{
        $E\ \subset\ \mathbb{R}$ est \underline{ouvert}, si pour tout $a\ \in\ E$ il existe $r >0$ tel que $] a-r, a+r [\ \subset E$\\
                \underline{L'intérieur $\overset{\circ}{E}$ de E} est le plus grand ensemble ouvert contenu dans $E$
    }
\end{note}

\begin{note}
    \xfield{
        Définir ce que signifie si un ensemble est fermé, et ce qu'est son adhérence.
    }
    \xfield{
        $E\ \subset\ \mathbb{R}$ est \underline{fermé}, si $E^c \equiv\ \mathbb{R}\ \backslash E$ est ouvert\\
        \underline{L'adhérence $\overline{E}$ de E} est le plus grand sous ensemble fermé de $\mathbb{R}$ qui contient $E$. \\
        On a $\overline{E}\ =\ \mathbb{R}\backslash \overset{\circ}{(\mathbb{R}\backslash E)}$ ou encore $\overline{E}\ = \{ a \in \mathbb{R}\ :\ \forall r \> 0,\ ] a-r, a+r[ \cap E \neq \varnothing\} $
    }
\end{note}

\begin{note}
    \xfield{ Définir ce qu'est le bord $\partial E\ de\ E$}
    \xfield{
        \underline{Le bord $\partial E\ de\ E$} :\\
        On a $\partial E\ =\ \overline E \backslash\ \overset{\circ}{E}$ ou encore $\partial E\ =\ \{ a\ \in\ \mathbb{R}\ :\ \forall r\ \> 0,\ ] a-r,\ a+r[\ \wedge\ E\ \neq\ \varnothing ,\ ] a-r,\ a+r[\ \cap\ (\mathbb{R}\ \backslash E \}$
    }
\end{note}

\begin{note}
    \xfield{Définir les points isolés et les points limites d'un ensemble.}
    \xfield{
        \underline{Point isolé}\\ $a\ \in\ E$ est un point isolé de $E$ s'il existe $r > 0$, tel que $]a-r,\ a+r[\ \cap\ E\ =\ \{a\}$\\
        \underline{Points limites}\\
        $\text{\{points limites\}}\ =\ \overline{E}\ \backslash\ \text{\{points isolés\}}\ =\ \{\ a\ \in\ \mathbb{R}\ :\ \forall r >0,\ ]a-r,\ a+r[\ \cap\ (E\ \backslash\ \{a\})\ \neq\ \varnothing \}$
    }
\end{note}

\begin{note}
    \xfield{Définir ce qu'est une suite et décrire sa notation.}
    \xfield{
        \underline{Définition} On appelle suite de nombres réels toute application $f\ :\ \mathbb{N}\ \rightarrow\ \mathbb{R}$.\\
        \underline{Notation} On pose $a_n\ =\ f(n)$ et on écrit $(a_n)$ ou $(a_n)_{n \ge 0}$ ou $a_0,a_1,$... pour la suite.
    }
\end{note}

\begin{note}
    \xfield{
        Comment définit-on une suite par récurrence ?\\
        Montrer comment l'appliquer sur la suite harmonique ($a_n\ =\ \frac{1}{n},\ n\ \in\ \mathbb{N})$
    }
    \xfield{
        \underline{Définition} Soit $a_1\ \in\ \mathbb{R}$, une fonction $g\ :\ \mathbb{R}\ \rightarrow\ \mathbb{R}$\\
        $a_n\ =\ g(a_{n-1})\ \ \ \ \ n\ =\ 2,3,4,...$\\
        \underline{Exemple : }\\
        $g(x)\ =\ \frac{x}{1+x}$ pour $a_1\ =\ 1$ la suite harmonique\\
        $a_2\ =\ g(a_1)\ =\ g(1)\ =\ \frac{1}{1+1}\ =\ \frac{1}{2}$\\
        $g(a_{n-1})\ =\ g(\frac{1}{n-1})\ =\ \frac{\frac{1}{n-1}}{1+ \frac{1}{n-1}}\ =\ \frac{1}{n}\ =\ a_n$}
\end{note}

\begin{note}
    \xfield{
        Définir la signification de :
        Une suite croissante
        Une suite décroissant
        Une suite monotone
    }
    \xfield{
        \underline{Suite croissante :} Une suite $(a_n)$ est croissante, si $a_{n+1}\ \ge\ a_n,\ \forall n\ \in\ \mathbb{N}$\\
        \underline{Suite décroissante :} Une suite $(a_n)$ est décroissante, si $a_{n+1}\ \le\ a_n,\ \forall n\ \in\ \mathbb{N}$\\
                \underline{Suite monotone :} Une suite $(a_n)$ est monotone, si elle est soit croissante, soit décroissante.}
\end{note}

\begin{note}
    \xfield{
        <u>Définir la signification de :</u>
        Une suite majorée
        Une suite minorée
        Une suite bornée
        <u>Et Définir :</u>
        Le plus petit majorant d'une suite
        Le plus grand minorant d'une suite
        Le minimum et le maximum d'une suite
    }
    \xfield{
        \underline{Suite majorée :} Une suite $(a_n)$ est majorée si $E\ =\ \{a_1,a_2,...\}, \subset\ \mathbb{R}$ est majoré\\
        \underline{Suite minorée :} Une suite $(a_n)$ est minorée si $E\ =\ \{a_1,a_2,...\}, \subset\ \mathbb{R}$ est minoré\\
        \underline{Suite bornée :} Une suite $(a_n)$ est bornée si elle est minorée \underline{et} majorée\\
        \underline{Le plus petit majorant d'une suite :} $sup(a_n)\ :=\ sup\{a_1,a_2,...\}$\\
        \underline{Le plus grand minorant d'une suite :} $inf(a_n)\ :=\ inf\{a_1,a_2,...\}$\\
        \underline{Le minimum et le maximum d'une suite :}$max(a_n)\ :=\ max\{a_1,a_2,...\}$ et $min(a_n)\ :=\ min\{a_1,a_2,...\}$ s'ils existent
    }
\end{note}

\begin{note}
    \xfield{Définir la limite d'une suite}
    \xfield{
        Une suite $(a_n)$ est convergente et admet pour limite (ou converge vers) $a\ \in\ \mathbb{R}$, et l'on écrit :\\
        \begin{center}
        $\lim\limits_{n \to \infty}\ a_n\ =\ a$
        \end{center}
        si pour tout $\epsilon >0$ il existe $n_0$ tel que $|a_n\ -\ a| < \epsilon, \forall n \ge n_0$
    }
\end{note}

\begin{note}
    \xfield{$(a+b)^3\ =\ ?$}
    \xfield{$(a+b)^3\ =\ a^3 + 3a^2b+3ab^2+b^3$}
\end{note}

\begin{note}
    \xfield{$(a-b)^3\ =\ ?$}
    \xfield{$(a-b)^3\ =\ a^3 - 3a^2b+3ab^2-b^3$}
\end{note}

\begin{note}
    \xfield{$a^3-b^3\ =\ ?$}
    \xfield{$a^3-b^3\ =\ (a-b)(a^2+ab+b^2)$}
\end{note}

\begin{note}
    \xfield{$a^3+b^3\ =\ ?$}
    \xfield{$a^3+b^3\ =\ (a+b)(a^2-ab+b^2)$}
\end{note}

\begin{note}
    \xfield{$(a+b+c)^2\ =\ ?$}
    \xfield{$(a+b+c)^2\ =\ a^2+b^2+c^2+2ab+2bc+2ac$}
\end{note}

\begin{note}
    \xfield{$\log_a(xy)\ =\ ?$}
    \xfield{$\log_a(xy)\ =\ \log_a(x)+\log_a(y)$}
\end{note}

\begin{note}
    \xfield{$\log_a(\frac{x}{y})\ =\ ?$}
    \xfield{$\log_a(\frac{x}{y})\ =\ \log_a(x)-\log_a(y)$}
\end{note}

\begin{note}
    \xfield{$\log_a(\frac{1}{y})\ =\ ?$}
    \xfield{$\log_a(\frac{1}{y})\ =\ -\log_a(y)$}
\end{note}

\begin{note}
    \xfield{$\log_a(x^p)\ =\ ?$}
    \xfield{$\log_a(x^p)\ =\ p \log_a(x)$}
\end{note}

\begin{note}
    \xfield{$|a+b|$ ? (inegalite du triangle)}
    \xfield{$ |a+b| \leqslant |a| + |b|\ $}
\end{note}

\begin{note}
    \xfield{$|a-b|$ ? (inegalite du triangle)}
    \xfield{$|a - b| \geqslant ||a|-|b||$}
\end{note}

\begin{note}
    \xfield{$\sum_{i=0}^n i\ =\ ? $}
    \xfield{$\sum_{i=0}^n i\ = \sum_{i=1}^n i\ =\ \frac{n(n+1)}{2}$}
\end{note}

\begin{note}
    \xfield{$\sum_{i=0}^n i^2 = ?$}
    \xfield{$\sum_{i=0}^n i^2 = \frac{n(n+1)(2n+1)}{6}$}
\end{note}

\begin{note}
    \xfield{
        $\sum_{i=0}^n a^i = ? $\\        
        $\sum_{i=0}^{n-1} a^i = ? $
    }
    \xfield{$\sum_{i=0}^n a^i = \frac{a^{n+1}-1}{a-1}$\\$\sum_{i=0}^{n-1} a^i = \frac{1-a^{n}}{1-a}$}
\end{note}

\begin{note}
    \xfield{Comment savoir si le polynome $P(x)$ est divisible par $x - a$ ? }
    \xfield{$P(x)$ est divisible par $x-a$ si $P(a) = 0$}
\end{note}

\begin{note}
    \xfield{$\cos^2(\alpha) + \sin^2(\alpha) = ?$}
    \xfield{$\cos^2(\alpha) + \sin^2(\alpha) = 1$}
\end{note}

\begin{note}
    \xfield{$\sin(\alpha \pm \beta) =$ ?}
    \xfield{$\sin(\alpha \pm \beta) = \sin \alpha \cos \beta \pm \cos \alpha \sin \beta$}
\end{note}

\begin{note}
    \xfield{$\cos(\alpha \pm \beta) =$ ?}
    \xfield{$\cos(\alpha \pm \beta) = \cos \alpha \cos \beta \mp \sin \alpha \sin \beta$ (Attention, au $\mp$ dans la dernière égalité )}
\end{note}

\begin{note}
    \xfield{$\sin 2\theta =$ ?}
    \xfield{$\sin 2\theta = 2 \sin \theta \cos \theta $}
\end{note}

\begin{note}
    \xfield{$\cos 2\theta =$ ?}
    \xfield{$\cos 2\theta = \cos^2 \theta - \sin^2 \theta \ = 2 \cos^2 \theta - 1\ = 1 - 2 \sin^2 \theta$}
\end{note}

\begin{note}
    \xfield{$\cos \theta \cos \varphi =$ ?}
    \xfield{$\cos \theta \cos \varphi = \frac{\cos(\theta - \varphi) + \cos(\theta + \varphi)} {2}$}
\end{note}

\begin{note}
    \xfield{$\sin \theta \sin \varphi =$ ?}
    \xfield{$\sin \theta \sin \varphi = \frac{\cos(\theta - \varphi) - \cos(\theta + \varphi)} {2}$}
\end{note}

\begin{note}
    \xfield{$\sin \theta \cos \varphi =$ ?}
    \xfield{$\sin \theta \cos \varphi = \frac{\sin(\theta + \varphi) + \sin(\theta - \varphi)} {2}$}
\end{note}

\begin{note}
    \xfield{$\cos \theta \sin \varphi =$ ?}
    \xfield{$\cos \theta \sin \varphi = \frac{\sin(\theta + \varphi) - \sin(\theta - \varphi)} {2}$}
\end{note}

\begin{note}
    \xfield{$\sin \theta \pm \sin \varphi =$ ?}
    \xfield{$\sin \theta \pm \sin \varphi = 2 \sin\left( \frac{\theta \pm \varphi}{2} \right) \cos\left( \frac{\theta \mp \varphi}{2} \right)$ (notice the $\mp$ !)}
\end{note}

\begin{note}
    \xfield{$\cos \theta + \cos \varphi =$ ?}
    \xfield{$\cos \theta + \cos \varphi = 2 \cos\left( \frac{\theta + \varphi} {2} \right) \cos\left( \frac{\theta - \varphi}{2} \right)$}
\end{note}

\begin{note}
    \xfield{$\cos \theta - \cos \varphi =$ ?}
    \xfield{$\cos \theta - \cos \varphi = -2\sin\left( \frac{\theta + \varphi} {2}\right) \sin\left(\frac {\theta - \varphi}{2}\right)$}
\end{note}

\begin{note}
    \xfield{$\sinh x =$ ?}
    \xfield{$\sinh x = \frac {e^x - e^{-x}} {2}$}
\end{note}

\begin{note}
    \xfield{$\cosh x = $ ?}
    \xfield{$\cosh x = \frac {e^x + e^{-x}} {2}$}
\end{note}

\begin{note}
    \xfield{$\tanh x =$ ?}
    \xfield{$\tanh x = \frac{\sinh x}{\cosh x} = \frac {e^x - e^{-x}} {e^x + e^{-x}}$}
\end{note}

\begin{note}
    \xfield{$\text{Si }\lim_{x \to c} f(x) = L_1 \text{ et }\lim_{x \to c} g(x) = L_2 \text{ alors:}$ \\ $\lim_{x \to c} \, [f(x) \pm g(x)] = $ ? \\$\lim_{x \to c} \, [f(x)g(x)] =$ ? \\ $\lim_{x \to c} \frac{f(x)}{g(x)} =$ ?\\ $\lim_{x \to c} \, f(x)^n =$ ?\\ $\lim_{x \to c} \, f(x)^\frac {1} {n} =$ ?}
    \xfield{$\lim_{x \to c} \, [f(x) \pm g(x)] = L_1 \pm L_2$\\$\lim_{x \to c} \, [f(x)g(x)] = L_1 \times L_2$ \\ $\lim_{x \to c} \frac{f(x)}{g(x)} = \frac{L_1}{L_2} \qquad \text{ if } L_2 \ne 0$ \\ $\lim_{x \to c} \, f(x)^n = L_1^n \qquad \text{ if }n \text{ is a positive integer}$ \\ $\lim_{x \to c} \, f(x)^\frac {1} {n} = L_1^\frac {1}{n} \qquad \text{ if }n \text{ is a positive integer, and if } n \text{ is even, then } L_1 > 0$}
\end{note}

\begin{note}
    \xfield{$\lim_{x\to1}\frac{\ln(x)}{x-1}=$ ? \\$\mbox{For } a > 1: \,$\\$\lim_{x \to 0^+} \log_a x = $ ?\\$\lim_{x \to \infty} \log_a x = $ ?\\$\lim_{x \to -\infty} a^x = $ ?\\$\mbox{If } a < 1: \,$\\$\lim_{x \to -\infty} a^x =$ ?}
    \xfield{$\lim_{x\to1}\frac{\ln(x)}{x-1}=1$\\$\mbox{For } a > 1: \,$\\$\lim_{x \to 0^+} \log_a x = -\infty$\\$\lim_{x \to \infty} \log_a x = \infty$\\$\lim_{x \to -\infty} a^x = 0$\\$\mbox{If } a < 1: \,$\\$\lim_{x \to -\infty} a^x = \infty$}
\end{note}

\begin{note}
    \xfield{$\lim_{x \to a} \sin x =$ ?\\$\lim_{x \to a} \cos x =$\\If $x$ is expressed in radians:\\$\lim_{x \to 0} \frac{\sin x}{x} =$ ?\\$\lim_{x \to 0} \frac{\tan x}{x} = $ ?\\$\lim_{x \to 0} \frac{1-\cos x}{x} =$ ?\\$\lim_{x \to 0} \frac{1-\cos x}{x^2} =$ ?}
    \xfield{$\lim_{x \to a} \sin x = \sin a$\\$\lim_{x \to a} \cos x = \cos a$\\If $x$ is expressed in radians:\\$\lim_{x \to 0} \frac{\sin x}{x} = 1$\\$\lim_{x \to 0} \frac{\tan x}{x} = 1$\\$\lim_{x \to 0} \frac{1-\cos x}{x} = 0$\\$\lim_{x \to 0} \frac{1-\cos x}{x^2} = \frac{1}{2}$}
\end{note}

\begin{note}
    \xfield{Comment peut on calculer une limite avec une racine ? Par exemple \\ $\lim_{x \to \infty} \sqrt{n+4} - \sqrt{n}$}
    \xfield{On multiplie par binôme conjugué.\\dans notre exemple :\\ $\lim_{x  \to \infty} \sqrt{n+4} - \sqrt{n} = \lim_{x \to \infty} \sqrt{n+4} - \sqrt{n} \cdot \frac{\sqrt{n+4} + \sqrt{n}}{\sqrt{n+4} + \sqrt{n}} = \lim_{x \to \infty} \frac{4}{\sqrt{n+4} + \sqrt{n}}$ qui est bien plus simple à cacluler}
\end{note}

\begin{note}
    \xfield{Quelle est le critère de convergence d'une suite ? Et son corrolaire ?}
    \xfield{
        Toute suise croissante et majorée (décroissante et minorée) est convergente et $\lim_{n \to \infty} a_n = sup(a_n)$ ($\lim_{n \to \infty} a_n = inf(a_n)$) \\
        Et son corrolaire : \\
        Toute suite monotone et bornée est convergente
    }
\end{note}

\begin{note}
    \xfield{Qu'est qu'une suite de Cauchy ? (1 définition + 1 condition)}
    \xfield{
        Une suite ($a_n$, $a_n \in \mathbb{R}$) est une suite de Cauchy, si pour tout $\epsilon > 0$ il existe $n_0 \in \mathbb{N}$, tel que pour tout $n,m \ge n_0$, $ | a_n - a_m | < \epsilon$\\
        Aussi, une suite $a_n$ de \underline{nombres réels} est de Cauchy, si et seulement si $a_n$ est une suite convergente
    }
\end{note}

\begin{note}
    \xfield{Donner le critère d'Alembert et le critère de Cauchy pour la convergence d'une série.}
    \xfield{
        Si\\
        $\lim_{n \to \infty} \left|\frac{a_{n+1}}{a_n}\right| = r$ existe (d'Alembert)\\
        Ou si\\
        $\lim_{n \to \infty} \left|a_n\right|^{\frac{1}{n}} = r$ existe (Cauchy)\\
        Si $r  1$, alors la serie diverge. So $r = 1$, il n'y a pas de conclusion possible en utilisant cette méthode.
    }
\end{note}

\begin{note}
    \xfield{
        Définir la parité d'une fonction.\\
        Expliquer comment elles se comportent quand elles sont composées.
    }
    \xfield{
        Premièrement on ne peut parler de parité que si $D(f)$ est symétrique.\\
        Si une fonction est paire, alors $f(-x) = f(x)$, par exemple : $f(x) = 0,1 , x^2, cos(x)$\\
        Si une fonction est impaire, alors $f(-x) = -f(x)$, par exemple : $f(x) = 0,x,x^3,sin(x)$\\
        La composée de deux fonctions impaires est impaire\\
        La composée $g \circ f$ d'une fonction paire $g$ avec une fonction impaire $f$ est une fonction paire.\\
        La composée $g \circ f$ d'une fonction quelconque g avec une fonction paire f est une fonction paire.}
\end{note}

\end{document}
