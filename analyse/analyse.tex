% -*- coding-system:utf-8 
% LATEX PREAMBLE --- needs to be imported manually
\documentclass[12pt]{article}
\special{papersize=3in,5in}
\usepackage[utf8]{inputenc}
\usepackage{amssymb,amsmath}
\pagestyle{empty}
\setlength{\parindent}{0in}

%%% commands that do not need to imported into Anki:
\usepackage{mdframed}
\newcommand*{\xfield}[1]{\begin{mdframed}\centering #1\end{mdframed}\bigskip}
\newenvironment{note}{}{}
\newcommand*{\tags}[1]{\paragraph{tags: }#1} 
% END OF THE PREAMBLE
\begin{document}

\begin{note}
	\xfield{Calculer $sin(\frac{\pi}{4})$ et $cos(\frac{\pi}{4})$}
	\xfield{On peut calculer facilement ces valeurs en prenant un triangle isocèle de côtés isocèles 1, il aura deux angles de $\frac{\pi}{4}$, un angle droit et une hypothénuse de $\sqrt{2}$ alors : \\
	$\sin(\frac{\pi}{4}) = \frac{1}{\sqrt{2}} = \frac{\sqrt{2}}{2}$\\
	$\cos(\frac{\pi}{4}) = \frac{\sqrt{2}}{2}$}
\end{note}
\begin{note}
	\xfield{Calculer
$sin(\frac{\pi}{6})$,  $cos(\frac{\pi}{6})$ , $sin(\frac{\pi}{3})$ et $cos(\frac{\pi}{3})$}
	\xfield{On peut calculer facilement ces valeurs en prenant un triangle équilatéral de côté 1, il aura alors des angles de $\frac{\pi}{3}$ et si l'on coupe le triangle en deux, cela formera un triangle rectangle avec un angle de $\frac{\pi}{6}$ qui aura :
	\begin{itemize}
	\item une hypothénuse de 1
	\item  un côté adjacent à $\frac{\pi}{3}$ et opposé à $\frac{\pi}{6}$ de $\frac{1}{2}$
	\item un côté opposé à $\frac{\pi}{3}$ et adjacent à $\frac{\pi}{6}$ de $\sqrt{(1^2-(\frac{1}{2})^2} = \frac{\sqrt{3}}{2}$
	\end{itemize}
	On a donc :\\
	$sin(\frac{\pi}{6}) = \frac{1}{2}$, $cos(\frac{\pi}{6}) = \frac{\sqrt{3}}{2}$ \\
	$sin(\frac{\pi}{3}) = \frac{\sqrt{3}}{2}$, $cos(\frac{\pi}{3}) = \frac{1}{2}$}
\end{note}
\begin{note}
	\xfield{Quelles sont les trois règles définissant une classe d'équivalence ?}
	\xfield{Relation d'équivalence sur $X$ notée $\sim$ \\
	R1) $x \sim x\ \ \forall x \in X$ (réflexive)\\
	R2) $x \sim y \rightarrow y\sim x$ (symétrique)\\
	R3) $x \sim y,y \sim z \rightarrow x \sim z$ (transitive)}
\end{note}
\begin{note}
	\xfield{Expliquer ce qu'est la représentation privilégiée d'un nombre rationnel et ce pour quoi elle peut-être utile.}
	\xfield{Le représentantant privilégié d'un nombre rationnel $x \in \mathbb{Q}$ est $\frac{p}{q}$ avec $q > 0$ et pgcd$(\lvert p \lvert,q) = 1$\\
	Soit $x = \frac{a}{1}$, $y = \frac{b}{1}$ alors $x+y = \frac{a+b}{1}$ et $x \cdot y = \frac{a\cdot b}{1}$\\
	Cette représentation peut être utiler pour montrer qu'un nombre n'est pas un rationnel.}
\end{note}
\begin{note}
	\xfield{Définir le raisonnement par récurrence (principe d'induction)}
	\xfield{\begin{enumerate}
	\item si $P(n_0)$ est vrai pour $n_0 \in \mathbb{N}$ (initialisation)
	\item et si pour tout $n \ge n_0$ $P(n) \rightarrow P(n+1)$ (le pas d'induction)
	\end{enumerate}
	alors $P(n)$ est vrai pour tout $n \ge n_0$}
\end{note}
\begin{note}
	\xfield{Définir le produit cartésien}
	\xfield{Soit $X,Y$ des ensembles \\
	$X \times Y = \{ (x,y) : x \in X, y \in Y \} $\\
	Exemple : $X = \{ 1,2 \} , Y = \{ 3,4 \} $\\
	$X \times Y = \{ (1,3),(1,4),(2,3),(2,4)\}$ \\
	attention : $X \times Y \neq Y \times X$ en général.}
\end{note}
\begin{note}
    \xfield{
        Définir ce qu'est un minorant et un majorant pour un ensemble.\\
        Définir ce que cela signifie si un ensemble est minoré ou majoré.\\
        Finalement, définir ce que cela signifie si un ensemble est borné.}
    \xfield{
        \underline{minorant :} $a\ \in \mathbb{R}$ est un minorant de $A\ \subset\ \mathbb{R},\ A\ \neq\ \varnothing$ si $a\ \le\ x,\ \forall x\ \in\ A$.\\        
        \underline{majorant :} $a\ \in \mathbb{R}$ est un majorant de $A\ \subset\ \mathbb{R},\ A\ \neq\ \varnothing$ si $a\ \ge\ x,\ \forall x\ \in\ A$.\\
        \underline{minoré ou borné inférieurement:} $A\ \subset\ \mathbb{R}$ est minoré si $A$ admet un minorant.\\
        \underline{majoré ou borné supérieurement:} $A\ \subset\ \mathbb{R}$ est majoré si $A$ admet un majorant.\\
        \underline{borné :} $A\ \subset\ \mathbb{R},\ A\ \neq\ \varnothing$ est borné Si $A$ est majoré et minoré.}
\end{note}

\begin{note}
    \xfield{
        Définir ce qu'est l'infimum d'un ensemble.\\
        Définir ce qu'est le minimum d'un ensemble.\\
    }
    \xfield{
        \underline{Infimum : } un minorant de $a \in\ \mathbb{R}$ est appelé infimum ou borne inférieure de $A\ \subset\ \mathbb{R},\ A\ \neq\ \varnothing$, noté :\\        
        \begin{center}
              $a\ =\ inf(A)$,\\
        \end{center}
        Si $A$ est le plus grand minorant de A, c'est-à-dire, si tout minorant b de A satifsait b $\le$ a\\    
        \underline{Minimum : }\\
        $si\ a=inf(A)\ \in A,\ alors\ inf(A)\ =:\ min(A)$ }
\end{note}

\begin{note}
    \xfield{
        Définir ce qu'est le supremum d'un ensemble.
        Définir ce qu'est le maximum d'un ensemble.
    }
    \xfield{
        \underline{supremum : } un majorant de $a \in\ \mathbb{R}$ est appelé supremum ou borne supérieure de $A\ \subset\ \mathbb{R},\ A\ \neq\ \varnothing$, noté : \\
                $a\ =\ sup(A)$,\\
        Si $A$ est le plus petit majorant de A, c'est-à-dire, si tout majorant b de A satifsait b $\ge$ a.\\        
        \underline{Maximum : }\\
        $si\ a=sup(A)\ \in A,\ alors\ sup(A)\ =:\ max(A)$.
    }
\end{note}

\begin{note}
    \xfield{Définir ce que signifie si un ensemble est ouvert, et ce qu'est son intérieur.}
    \xfield{
        $E\ \subset\ \mathbb{R}$ est \underline{ouvert}, si pour tout $a\ \in\ E$ il existe $r >0$ tel que $] a-r, a+r [\ \subset E$\\
                \underline{L'intérieur $\overset{\circ}{E}$ de E} est le plus grand ensemble ouvert contenu dans $E$
    }
\end{note}

\begin{note}
    \xfield{
        Définir ce que signifie si un ensemble est fermé, et ce qu'est son adhérence.
    }
    \xfield{
        $E\ \subset\ \mathbb{R}$ est \underline{fermé}, si $E^c \equiv\ \mathbb{R}\ \backslash E$ est ouvert\\
        \underline{L'adhérence $\overline{E}$ de E} est le plus grand sous ensemble fermé de $\mathbb{R}$ qui contient $E$. \\
        On a $\overline{E}\ =\ \mathbb{R}\backslash \overset{\circ}{(\mathbb{R}\backslash E)}$ ou encore $\overline{E}\ = \{ a \in \mathbb{R}\ :\ \forall r \> 0,\ ] a-r, a+r[ \cap E \neq \varnothing\} $
    }
\end{note}

\begin{note}
    \xfield{ Définir ce qu'est le bord $\partial E\ de\ E$}
    \xfield{
        \underline{Le bord $\partial E\ de\ E$} :\\
        On a $\partial E\ =\ \overline E \backslash\ \overset{\circ}{E}$ ou encore $\partial E\ =\ \{ a\ \in\ \mathbb{R}\ :\ \forall r\ \> 0,\ ] a-r,\ a+r[\ \wedge\ E\ \neq\ \varnothing ,\ ] a-r,\ a+r[\ \cap\ (\mathbb{R}\ \backslash E \}$
    }
\end{note}

\begin{note}
    \xfield{Définir les points isolés et les points limites d'un ensemble.}
    \xfield{
        \underline{Point isolé}\\ $a\ \in\ E$ est un point isolé de $E$ s'il existe $r > 0$, tel que $]a-r,\ a+r[\ \cap\ E\ =\ \{a\}$\\
        \underline{Points limites}\\
        $\text{\{points limites\}}\ =\ \overline{E}\ \backslash\ \text{\{points isolés\}}\ =\ \{\ a\ \in\ \mathbb{R}\ :\ \forall r >0,\ ]a-r,\ a+r[\ \cap\ (E\ \backslash\ \{a\})\ \neq\ \varnothing \}$
    }
\end{note}

\begin{note}
	\xfield{Comment exprimer $z=a-ib$ en forme polaire.
	\\exprimer $1-\sqrt{3}i$ en forme polaire}
	\xfield{La forme polaire de $z$ s'écrit comme $z= \vert z \vert e^{i \varphi}$ \\
	où $\vert z \vert = \sqrt{a^2+b^2}$\\
	et $\varphi = 2 \cdot \arctan(\frac{b}{a+\vert z \vert})$ si $y \neq 0$ ou si $x > 0 et y = 0$, sinon $\varphi = \pi$\\
	Pour $z = 1-\sqrt{3}i$\\
	$\vert z \vert = \sqrt{1^2+\sqrt{3}^2} = 2$\\
	et $\varphi =  2 \cdot \arctan(\frac{\sqrt{3}}{1+2}) = 2 \cdot = \arctan(\frac{1}{\sqrt{3}})=2 \cdot \frac{\pi}{6} = \frac{\pi}{3}$\\
	donc $z= 2 e^{i \frac{\pi}{3}}$}
\end{note}

\begin{note}
    \xfield{Définir ce qu'est une suite et décrire sa notation.}
    \xfield{
        \underline{Définition} On appelle suite de nombres réels toute application $f\ :\ \mathbb{N}\ \rightarrow\ \mathbb{R}$.\\
        \underline{Notation} On pose $a_n\ =\ f(n)$ et on écrit $(a_n)$ ou $(a_n)_{n \ge 0}$ ou $a_0,a_1,$... pour la suite.
    }
\end{note}

\begin{note}
    \xfield{
        Comment définit-on une suite par récurrence ?\\
        Montrer comment l'appliquer sur la suite harmonique ($a_n\ =\ \frac{1}{n},\ n\ \in\ \mathbb{N})$
    }
    \xfield{
        \underline{Définition} Soit $a_1\ \in\ \mathbb{R}$, une fonction $g\ :\ \mathbb{R}\ \rightarrow\ \mathbb{R}$\\
        $a_n\ =\ g(a_{n-1})\ \ \ \ \ n\ =\ 2,3,4,...$\\
        \underline{Exemple : }\\
        $g(x)\ =\ \frac{x}{1+x}$ pour $a_1\ =\ 1$ la suite harmonique\\
        $a_2\ =\ g(a_1)\ =\ g(1)\ =\ \frac{1}{1+1}\ =\ \frac{1}{2}$\\
        $g(a_{n-1})\ =\ g(\frac{1}{n-1})\ =\ \frac{\frac{1}{n-1}}{1+ \frac{1}{n-1}}\ =\ \frac{1}{n}\ =\ a_n$}
\end{note}

\begin{note}
    \xfield{
        Définir la signification de :
        \begin{itemize}
        \item Une suite croissante
        \item Une suite décroissant
        \item Une suite monotone
        \end{itemize}
    }
    \xfield{
        \underline{Suite croissante :} Une suite $(a_n)$ est croissante, si $a_{n+1}\ \ge\ a_n,\ \forall n\ \in\ \mathbb{N}$\\
        \underline{Suite décroissante :} Une suite $(a_n)$ est décroissante, si $a_{n+1}\ \le\ a_n,\ \forall n\ \in\ \mathbb{N}$\\
                \underline{Suite monotone :} Une suite $(a_n)$ est monotone, si elle est soit croissante, soit décroissante.}
\end{note}

\begin{note}
    \xfield{
        \begin{itemize}
        \item Définir la signification de :
        \item Une suite majorée
        \item Une suite minorée
        \item Une suite bornée
        \end{itemize}
        Et Définir :
        \begin{itemize}
        \item Le plus petit majorant d'une suite
        \item Le plus grand minorant d'une suite
        \item Le minimum et le maximum d'une suite
        \end{itemize}
    }
    \xfield{
        \underline{Suite majorée :} Une suite $(a_n)$ est majorée si $E\ =\ \{a_1,a_2,...\}, \subset\ \mathbb{R}$ est majoré\\
        \underline{Suite minorée :} Une suite $(a_n)$ est minorée si $E\ =\ \{a_1,a_2,...\}, \subset\ \mathbb{R}$ est minoré\\
        \underline{Suite bornée :} Une suite $(a_n)$ est bornée si elle est minorée \underline{et} majorée\\
        \underline{Le plus petit majorant d'une suite :} $sup(a_n)\ :=\ sup\{a_1,a_2,...\}$\\
        \underline{Le plus grand minorant d'une suite :} $inf(a_n)\ :=\ inf\{a_1,a_2,...\}$\\
        \underline{Le minimum et le maximum d'une suite :}$max(a_n)\ :=\ max\{a_1,a_2,...\}$ et $min(a_n)\ :=\ min\{a_1,a_2,...\}$ s'ils existent
    }
\end{note}

\begin{note}
    \xfield{Définir la limite d'une suite}
    \xfield{
        Une suite $(a_n)$ est convergente et admet pour limite (ou converge vers) $a\ \in\ \mathbb{R}$, et l'on écrit :\\
        \begin{center}
        $\lim\limits_{n \to \infty}\ a_n\ =\ a$
        \end{center}
        si pour tout $\epsilon >0$ il existe $n_0$ tel que $|a_n\ -\ a| < \epsilon, \forall n \ge n_0$
    }
\end{note}

\begin{note}
    \xfield{$(a+b)^3\ =\ ?$}
    \xfield{$(a+b)^3\ =\ a^3 + 3a^2b+3ab^2+b^3$}
\end{note}

\begin{note}
    \xfield{$(a-b)^3\ =\ ?$}
    \xfield{$(a-b)^3\ =\ a^3 - 3a^2b+3ab^2-b^3$}
\end{note}

\begin{note}
    \xfield{$a^3-b^3\ =\ ?$}
    \xfield{$a^3-b^3\ =\ (a-b)(a^2+ab+b^2)$}
\end{note}

\begin{note}
    \xfield{$a^3+b^3\ =\ ?$}
    \xfield{$a^3+b^3\ =\ (a+b)(a^2-ab+b^2)$}
\end{note}

\begin{note}
    \xfield{$(a+b+c)^2\ =\ ?$}
    \xfield{$(a+b+c)^2\ =\ a^2+b^2+c^2+2ab+2bc+2ac$}
\end{note}

\begin{note}
    \xfield{$\log_a(xy)\ =\ ?$}
    \xfield{$\log_a(xy)\ =\ \log_a(x)+\log_a(y)$}
\end{note}

\begin{note}
    \xfield{$\log_a(\frac{x}{y})\ =\ ?$}
    \xfield{$\log_a(\frac{x}{y})\ =\ \log_a(x)-\log_a(y)$}
\end{note}

\begin{note}
    \xfield{$\log_a(\frac{1}{y})\ =\ ?$}
    \xfield{$\log_a(\frac{1}{y})\ =\ -\log_a(y)$}
\end{note}

\begin{note}
    \xfield{$\log_a(x^p)\ =\ ?$}
    \xfield{$\log_a(x^p)\ =\ p \log_a(x)$}
\end{note}

\begin{note}
    \xfield{$|a+b|$ ? (inegalite du triangle)}
    \xfield{$ |a+b| \leqslant |a| + |b|\ $}
\end{note}

\begin{note}
    \xfield{$|a-b|$ ? (inegalite du triangle)}
    \xfield{$|a - b| \geqslant ||a|-|b||$}
\end{note}

\begin{note}
    \xfield{$\sum_{i=0}^n i\ =\ ? $}
    \xfield{$\sum_{i=0}^n i\ = \sum_{i=1}^n i\ =\ \frac{n(n+1)}{2}$}
\end{note}

\begin{note}
    \xfield{$\sum_{i=0}^n i^2 = ?$}
    \xfield{$\sum_{i=0}^n i^2 = \frac{n(n+1)(2n+1)}{6}$}
\end{note}

\begin{note}
    \xfield{
        $\sum_{i=0}^n a^i = ? $\\        
        $\sum_{i=0}^{n-1} a^i = ? $
    }
    \xfield{$\sum_{i=0}^n a^i = \frac{a^{n+1}-1}{a-1}$\\$\sum_{i=0}^{n-1} a^i = \frac{1-a^{n}}{1-a}$}
\end{note}

\begin{note}
    \xfield{Comment savoir si le polynome $P(x)$ est divisible par $x - a$ ? }
    \xfield{$P(x)$ est divisible par $x-a$ si $P(a) = 0$}
\end{note}

\begin{note}
    \xfield{$\cos^2(\alpha) + \sin^2(\alpha) = ?$}
    \xfield{$\cos^2(\alpha) + \sin^2(\alpha) = 1$}
\end{note}

\begin{note}
    \xfield{$\sin(\alpha \pm \beta) =$ ?}
    \xfield{$\sin(\alpha \pm \beta) = \sin \alpha \cos \beta \pm \cos \alpha \sin \beta$}
\end{note}

\begin{note}
    \xfield{$\cos(\alpha \pm \beta) =$ ?}
    \xfield{$\cos(\alpha \pm \beta) = \cos \alpha \cos \beta \mp \sin \alpha \sin \beta$ (Attention, au $\mp$ dans la dernière égalité )}
\end{note}

\begin{note}
    \xfield{$\sin 2\theta =$ ?}
    \xfield{$\sin 2\theta = 2 \sin \theta \cos \theta $}
\end{note}

\begin{note}
    \xfield{$\cos 2\theta =$ ?}
    \xfield{$\cos 2\theta = \cos^2 \theta - \sin^2 \theta \ = 2 \cos^2 \theta - 1\ = 1 - 2 \sin^2 \theta$}
\end{note}

\begin{note}
    \xfield{$\cos \theta \cos \varphi =$ ?}
    \xfield{$\cos \theta \cos \varphi = \frac{\cos(\theta - \varphi) + \cos(\theta + \varphi)} {2}$}
\end{note}

\begin{note}
    \xfield{$\sin \theta \sin \varphi =$ ?}
    \xfield{$\sin \theta \sin \varphi = \frac{\cos(\theta - \varphi) - \cos(\theta + \varphi)} {2}$}
\end{note}

\begin{note}
    \xfield{$\sin \theta \cos \varphi =$ ?}
    \xfield{$\sin \theta \cos \varphi = \frac{\sin(\theta + \varphi) + \sin(\theta - \varphi)} {2}$}
\end{note}

\begin{note}
    \xfield{$\cos \theta \sin \varphi =$ ?}
    \xfield{$\cos \theta \sin \varphi = \frac{\sin(\theta + \varphi) - \sin(\theta - \varphi)} {2}$}
\end{note}

\begin{note}
    \xfield{$\sin \theta \pm \sin \varphi =$ ?}
    \xfield{$\sin \theta \pm \sin \varphi = 2 \sin\left( \frac{\theta \pm \varphi}{2} \right) \cos\left( \frac{\theta \mp \varphi}{2} \right)$ (notice the $\mp$ !)}
\end{note}

\begin{note}
    \xfield{$\cos \theta + \cos \varphi =$ ?}
    \xfield{$\cos \theta + \cos \varphi = 2 \cos\left( \frac{\theta + \varphi} {2} \right) \cos\left( \frac{\theta - \varphi}{2} \right)$}
\end{note}

\begin{note}
    \xfield{$\cos \theta - \cos \varphi =$ ?}
    \xfield{$\cos \theta - \cos \varphi = -2\sin\left( \frac{\theta + \varphi} {2}\right) \sin\left(\frac {\theta - \varphi}{2}\right)$}
\end{note}

\begin{note}
    \xfield{$\sinh x =$ ?}
    \xfield{$\sinh x = \frac {e^x - e^{-x}} {2}$}
\end{note}

\begin{note}
    \xfield{$\cosh x = $ ?}
    \xfield{$\cosh x = \frac {e^x + e^{-x}} {2}$}
\end{note}

\begin{note}
    \xfield{$\tanh x =$ ?}
    \xfield{$\tanh x = \frac{\sinh x}{\cosh x} = \frac {e^x - e^{-x}} {e^x + e^{-x}}$}
\end{note}

\begin{note}
    \xfield{$\text{Si }\lim_{x \to c} f(x) = L_1 \text{ et }\lim\limits_{x \to c} g(x) = L_2 \text{ alors:}$ \\ $\lim\limits_{x \to c} \, [f(x) \pm g(x)] = $ ? \\$\lim\limits_{x \to c} \, [f(x)g(x)] =$ ? \\ $\lim\limits_{x \to c} \frac{f(x)}{g(x)} =$ ?\\ $\lim\limits_{x \to c} \, f(x)^n =$ ?\\ $\lim\limits_{x \to c} \, f(x)^\frac {1} {n} =$ ?}
    \xfield{$\lim\limits_{x \to c} \, [f(x) \pm g(x)] = L_1 \pm L_2$\\$\lim\limits_{x \to c} \, [f(x)g(x)] = L_1 \times L_2$ \\ $\lim\limits_{x \to c} \frac{f(x)}{g(x)} = \frac{L_1}{L_2} \qquad \text{ if } L_2 \ne 0$ \\ $\lim\limits_{x \to c} \, f(x)^n = L_1^n \qquad \text{ if }n \text{ is a positive integer}$ \\ $\lim\limits_{x \to c} \, f(x)^\frac {1} {n} = L_1^\frac {1}{n} \qquad \text{ if }n \text{ is a positive integer, and if } n \text{ is even, then } L_1 > 0$}
\end{note}

\begin{note}
	\xfield{$\lim\limits_{x\to1}\frac{\ln(x)}{x-1}=$ ?}
	\xfield{$\lim\limits_{x\to1}\frac{\ln(x)}{x-1}=1$}
\end{note}

\begin{note}
	\xfield{$\lim\limits_{x\to \infty} \frac{ln(x)}{x^p}= $ ?}
	\xfield{$\lim\limits_{x\to \infty} \frac{ln(x)}{x^p}= 0$ si $p>0$}
\end{note}

\begin{note}
	\xfield{$\lim\limits_{x \to \infty}(1+\frac{a}{x})^x = $ ?}
	\xfield{$\lim\limits_{x \to \infty}(1+\frac{a}{x})^x = e^a$}
\end{note}


\begin{note}
	\xfield{$\lim\limits_{x \to \infty} \sqrt[x]{a} =$ ?}
	\xfield{$\lim\limits_{x \to \infty} \sqrt[x]{a} = 1$ for $a > 0$}
\end{note}

\begin{note}
	\xfield{$\mbox{Pour } a > 1:\ ,\ \lim\limits_{x \to 0^+} \log_a x = $ ?}
	\xfield{$\lim\limits_{x \to 0^+} \log_a x = -\infty$}
\end{note}

\begin{note}
	\xfield{$\mbox{Pour } a > 1:\ ,$ $\lim\limits_{x \to \infty} \log_a x = $ ?}
	\xfield{$\lim\limits_{x \to \infty} \log_a x = \infty$}
\end{note}

\begin{note}
	\xfield{$\mbox{Pour } a > 1:\ ,$ $\lim\limits_{x \to -\infty} a^x = $ ?}
	\xfield{$\lim\limits_{x \to -\infty} a^x = 0$}
\end{note}

\begin{note}
	\xfield{$\mbox{Pour } a < 1:\ ,$ $\lim\limits_{x \to -\infty} a^x =$ ?}
	\xfield{$\lim\limits_{x \to -\infty} a^x = \infty$}
\end{note}

\begin{note}
	\xfield{If $x$ is expressed in radians:\\
	$\lim\limits_{x \to 0} \frac{\sin x}{x} =$ ?}
	\xfield{$\lim\limits_{x \to 0} \frac{\sin x}{x} = 1$}
\end{note}

\begin{note}
	\xfield{If $x$ is expressed in radians:\\
	$\lim\limits_{x \to 0} \frac{\tan x}{x} = $ ?}
	\xfield{$\lim\limits_{x \to 0} \frac{\tan x}{x} = 1$}
\end{note}

\begin{note}
	\xfield{If $x$ is expressed in radians:\\
	$\lim\limits_{x \to 0} \frac{1-\cos x}{x} =$ ?}
	\xfield{$\lim\limits_{x \to 0} \frac{1-\cos x}{x} = 0$}
\end{note}

\begin{note}
	\xfield{If $x$ is expressed in radians:\\
	$\lim\limits_{x \to 0} \frac{1-\cos x}{x^2} =$ ?}
	\xfield{$\lim\limits_{x \to 0} \frac{1-\cos x}{x^2} = \frac{1}{2}$}
\end{note}

\begin{note}
    \xfield{Comment peut on calculer une limite avec une racine ? Par exemple \\ $\lim\limits_{x \to \infty} \sqrt{n+4} - \sqrt{n}$}
    \xfield{On multiplie par binôme conjugué.\\dans notre exemple :\\ $\lim\limits_{x  \to \infty} \sqrt{n+4} - \sqrt{n} = \lim\limits_{x \to \infty} \sqrt{n+4} - \sqrt{n} \cdot \frac{\sqrt{n+4} + \sqrt{n}}{\sqrt{n+4} + \sqrt{n}} = \lim\limits_{x \to \infty} \frac{4}{\sqrt{n+4} + \sqrt{n}}$ qui est bien plus simple à cacluler}
\end{note}

\begin{note}
    \xfield{Quelle est le critère de convergence d'une suite ? Et son corrolaire ?}
    \xfield{
        Toute suise croissante et majorée (décroissante et minorée) est convergente et $\lim\limits_{n \to \infty} a_n = sup(a_n)$ ($\lim\limits_{n \to \infty} a_n = inf(a_n)$) \\
        Et son corrolaire : \\
        Toute suite monotone et bornée est convergente
    }
\end{note}

\begin{note}
    \xfield{Qu'est qu'une suite de Cauchy ? (1 définition + 1 condition)}
    \xfield{
        Une suite ($a_n$, $a_n \in \mathbb{R}$) est une suite de Cauchy, si pour tout $\epsilon > 0$ il existe $n_0 \in \mathbb{N}$, tel que pour tout $n,m \ge n_0$, $ | a_n - a_m | < \epsilon$\\
        Aussi, une suite $a_n$ de \underline{nombres réels} est de Cauchy, si et seulement si $a_n$ est une suite convergente
    }
\end{note}

\begin{note}
	\xfield{Qu'est cela signifie si une série $\sum\limits^{\infty}_{k=1} a_k$ converge absolument ?}
	\xfield{Si $\sum\limits^{\infty}_{k=1} a_k$ converge absolument, alors $\sum\limits^{\infty}_{k=1} \vert a_k\vert$ converge }
\end{note}

\begin{note}
	\xfield{Donner le critère nécessaire à la converge de la série $\sum\limits^{\infty}_{k=1} a_k$}
	\xfield{$\sum\limits^{\infty}_{k=1} a_k$ $\Rightarrow$ $\lim\limits_{k\to \infty}a_k = 0$}
\end{note}

\begin{note}
	\xfield{Donner le critère de Leibniz pour la convergence d'une série alternée $a_k$}
	\xfield{Si $a_k$ est une série alternée ( $(-1)^k a_k \ge 0 ou \le 0$ pour tout $k$ ) et si $\vert a_k \vert$ est strictement décroissante ($\vert a_{k+1} \vert < \vert a_k \vert$ pour tout $k$) alors la série $\sum\limits^{\infty}_{k=1} a_k$ converge}
\end{note}

\begin{note}
    \xfield{Donner le critère d'Alembert et le critère de Cauchy pour la convergence d'une série.}
    \xfield{
        Si\\
        $\lim\limits_{n \to \infty} \left|\frac{a_{n+1}}{a_n}\right| = r$ existe (d'Alembert)\\
        Ou si\\
        $\lim\limits_{n \to \infty} \left|a_n\right|^{\frac{1}{n}} = r$ existe (Cauchy)\\
        Si $r < 1$ alors la série converge.\\
	Si $r > 1$, alors la serie diverge.\\
	Si $r = 1$, il n'y a pas de conclusion possible en utilisant cette méthode.
    }
\end{note}

\begin{note}
    \xfield{
        Définir la parité d'une fonction.\\
        Expliquer comment elles se comportent quand elles sont composées.
    }
    \xfield{
        Premièrement on ne peut parler de parité que si $D(f)$ est symétrique.\\
        Si une fonction est paire, alors $f(-x) = f(x)$, par exemple : $f(x) = 0,1 , x^2, cos(x)$\\
        Si une fonction est impaire, alors $f(-x) = -f(x)$, par exemple : $f(x) = 0,x,x^3,sin(x)$\\
        La composée de deux fonctions impaires est impaire\\
        La composée $g \circ f$ d'une fonction paire $g$ avec une fonction impaire $f$ est une fonction paire.\\
        La composée $g \circ f$ d'une fonction quelconque g avec une fonction paire f est une fonction paire.}
\end{note}

\begin{note}
	\xfield{\begin{itemize}
	\item Définir la limite (épointée) d'un fonction.
	\item Définir la limite (épointée) à droite et à gauche
	\item Définir la limite (épointée) avec $\epsilon$ et $\delta$
	\end{itemize} }
	\xfield{\begin{itemize}
	\item La fonction $f : \mathbb{R} \rightarrow \mathbb{R}$ admet pour limite $l \in \mathbb{R}$ lorsque $x$ tend vers $x_0$, si pour \underline{toute suite} (si ce n'est pas le cas, la limite n'existe pas) $x_n,\ x_n \in D(f) \smallsetminus \{x_0\},$ tel que $\lim\limits_{n \to \infty}x_n = x_0$ \\
	La suite $y_n,\ y_n = f(x_n)$ converge et $\lim\limits_{n \to \infty} = l$
	\item La fonction $f : \mathbb{R} \rightarrow \mathbb{R}$ admet pour limite à droit (à gauche) $l_+ \in \mathbb{R}$ ($l_- \in \mathbb{R}$) lorsque $x$ tend vers $x_0$, si pour \underline{toute suite} $x_n,\ x_n \in D(f),$ tel que $x_n > x_0$ ($x_n < x_0$) \\
	La suite $y_n,\ y_n = f(x_n)$ converge et $\lim\limits_{n \to \infty} = l_+$ ($\lim\limits_{n \to \infty} = l_-$)\\
	Notation :\\
	$\lim\limits_{x \to x_+^*}f(x)\lim\limits_{\substack{x \to x_0 \\ x > x_0}}f(x) = l_+$\\
	$\lim\limits_{x \to x_-^*}f(x)\lim\limits_{\substack{x \to x_0 \\ x < x_0}}f(x) = l_-$
	\item La fonction $f : \mathbb{R} \rightarrow \mathbb{R}$ admet pour limite $l \in \mathbb{R}$ lorsque $x$ tend vers $x_0$, si $\forall \epsilon > 0 \exists \delta$ tel que $|f(x) -l| < \epsilon,\ \forall x \in D(f)$ tels que $0< |x-x_0| < \delta$
	\end{itemize} }
\end{note}

\begin{note}
	\xfield{Donner le théorème des deux gendarmes pour trouver la limite de la fonction $h(x)$}
	\xfield{Soit $f,\ g,\ h$ des fonctions $\mathbb{R} \to \mathbb{R}$ telles que :
	\begin{enumerate}
		\item $\lim\limits_{\substack{x\to x_0\\x \neq x_0}}f(x) = \lim\limits_{\substack{x\to x_0\\x \neq x_0}}g(x) = l$
		\item pour $x$ proche de $x_0$ ($\exists \epsilon > 0, tel que \forall x \neq x_0,\ |x-x_0| < \epsilon$)\\
		$f(x) \le h(x) \le g(x)$
	\end{enumerate}
	alors  $\lim\limits_{\substack{x\to x_0\\x \neq x_0}}h(x) = l$}
\end{note}

\begin{note}
	\xfield{\begin{enumerate}
		\item Définir la continuité d'une fonction
		\item Définir la conitnuité à droit (à gauche) d'une fonction
	\end{enumerate} }
	\xfield{\begin{enumerate}
		\item La fonction $f:\mathbb{R}\to \mathbb{R}$ est continue en $x_0 \in I =\ ]a,b[\ \subset D(f)$ si $\lim\limits_{\substack{x\to x_0\\x \neq x_0}}f(x) = f(x_0)$
		\item La fonction $f:\mathbb{R}\to \mathbb{R}$ est continue à droite (à gauche) en $x_0 \in I =\ [a,b[\ (I =\ ]a,b])\ \subset D(f)$ si\\
		 $\lim\limits_{\substack{x\to x_0\\x > x_0}}f(x) = f(x_0)$ ($\lim\limits_{\substack{x\to x_0\\x < x_0}}f(x) = f(x_0)$)
	\end{enumerate} }
\end{note}

\begin{note}
	\xfield{\begin{enumerate}
		\item Définir la dérivabilité d'une fonction
		\item Définir la différentiabilité
		\item Définir la Dérivée d'une fonction
		\item Définir la dérivée d'ordre n
		\item Qu'implique la Dérivabilité
		\item Définir ce que veut dire dérivable à droite (à gauche)
	\end{enumerate} }
	\xfield{\begin{enumerate}
		\item La fonction $f$ est dérivable en $x_0$ si la limite :\\
			$\lim\limits_{\substack{x\to x_0\\x > x_0}} \frac{f(x)-f(x_0)}{x-x_0} \equiv d(x_0) \in \mathbb{R}$ existe
		\item La fonction $f$ est différentiable en $x_0$ s'il existe un nombre $a \in \mathbb{R}$ et une fonction $r:\ ]a,b[ \to \mathbb{R}$ tels que :\\
		$f(x_0 + h) = f(x_0) + a \cdot h + r(x_0 + h) \cdot h$\\
		avec $\lim\limits_{\substack{x\to x_0\\x > x_0}} r(x_0+h) = 0$
		\item Soit la fonction $f$ dérivable sur $]a,b[$, alors on peut définir $f'$, appelée la dérivée de $f$, par \\
		$f'(x) = d(x) \equiv \lim\limits_{h \to 0} \frac{f(x+h)-f(x)}{h}$
		\item La n-ième dérivée de $f$ : $f^{(n)} = (f^{(n-1)})'(x)$
		\item Une fonction qui est dérivable en $x_0$ est continue en $x_0$, toutefois, l'inverse n'est pas toujours vrai.
		\item La fonction $f : \mathbb{R} \rightarrow \mathbb{R}$ est dérivable à droite (à gauche) en $x_0 \in I$, $I = [a,b[$  ($I = ]a,b])$ $\subset D(f)$ si \\
		$\lim\limits_{\substack{h \to 0\\ h > 0}} \frac{f(x+h)-f(x)}{h} = d_+(x)$ existe.\\
		$\big(\lim\limits_{\substack{h \to 0\\ h < 0}} \frac{f(x+h)-f(x)}{h} = d_+(x)\big)$
	\end{enumerate} }
\end{note}


\begin{note}
	\xfield{Théorème de Rolle}
	\xfield{Soient $a$ et $b$ deux réels tels que $a < b$ et $f$ une fonction à valeurs réelles continue sur $[a, b]$ et dérivable sur $]a, b[$ telle que\\
$f(a)=f(b)$\\
Alors, il existe (au moins) un réel $c$ dans $]a, b[$ tel que\\
$f'(c)=0$ }
\end{note}

\begin{note}
	\xfield{Théorème des accroissements finis}
	\xfield{Soit $f : \mathbb{R} \rightarrow \mathbb{R}$, $[a,b] \subset D(f),\ b>a$, $f$ continue sur $[a,b]$, dérivable sur $]a,b[$. Alors il existe $u\in ]a,b[$ tel que\\
$f'(u)=\frac{f(b)-f(a)}{b-a}$}
\end{note}

\begin{note}
	\xfield{Théorème des acrroissements finis généralisé}
	\xfield{Soit $f,g : \mathbb{R} \rightarrow \mathbb{R}$, $[a,b] \subset D(f) \cap D(g),\ b>a$, $f,g$ continues sur $[a,b]$, dérivables sur $]a,b[$, $g'(x) \neq 0$. Alors il existe $u\in ]a,b[$ tel que\\
$\frac{f'(u)}{g'(u)}=\frac{f(b)-f(a)}{g(b)-g(a)}$}
\end{note}

\begin{note}
	\xfield{Théorème de Bernoulli de l'Hospital}
	\xfield{Soit $f,g$ deux fonctions dérivables sur $]a,b[ \subset D(f) \cap D(g)$ avec $g'(x) \neq 0$\\
	Si $\lim\limits_{\substack{x \to a\\ x > a}} f(x) = \lim\limits_{\substack{x \to a\\ x > a}} g(x) = 0$\\
	 et si \\
	 $\lim\limits_{\substack{x \to a\\ x > a}} \frac{f'(x)}{g'(x)} = l \in \mathbb{R}$ existe.\\
	 alors $\lim\limits_{\substack{x \to a\\ x > a}} \frac{f(x)}{g(x)} = \lim\limits_{\substack{x \to a\\ x > a}} \frac{f'(x)}{g'(x)} = l$
	 }
\end{note}

\begin{note}
	\xfield{Quelles information la dérivée nous donne sur le graphe d'une fonction ?}
	\xfield{\begin{itemize}
		\item $f' \ge 0$ sur $]a,b[$ $\Rightarrow$ $f$ croissant sur $[a,b]$
		\item $f' > 0$ sur $]a,b[$ $\Rightarrow$ $f$ strictement croissant sur $[a,b]$
		\item $f' \le 0$ sur $]a,b[$ $\Rightarrow$ $f$ décroissant sur $[a,b]$
		\item $f' < 0$ sur $]a,b[$ $\Rightarrow$ $f$ strictement décroissant sur $[a,b]$
		\item Soit $[a,b]\ \subset D(f),\ b > a,\ f$ continue sur $[a,b]$, dérivable sur $]a,b[$.\\
		Alors $f'=0$ sur $]a,b[ \Rightarrow f$ constante sur $[a,b]$
		\item Soit $f'(x_0)=0$, si $f''(x_0)<0$ f admet un maximum local en $x_0$,
		si $f''(x_0)>0$ f admet un minimum local en $x_0$
		\item Soit $f \mathbb{R} \to \mathbb{R}, [a,b] \subset D(f)$, $f$ continue sur $[a,b]$. Les points $x_0 \in [a,b]$ pour lesquels $f$ admet un extremum global sont éléments de : \begin{enumerate}
			\item \{a,b\}
			\item \{des points où $f'$ n'existe pas\}
			\item \{ les points où $f'=0$ \}
		\end{enumerate}
		\item Si $f'$ est croissante sur $I_0$ (en particulier si $f'' \ge 0$ sur $I_0$), alors f est convexe sur $I_0$
		\item Si $f'$ est décroissante sur $I_0$ (en particulier si $f'' \le 0$ sur $I_0$), alors f est concave sur $I_0$ 
		\item Si $f''(x_0) = 0$ alors $x_0$ est un point d'inflexion
	\end{itemize} }
\end{note}

\begin{note}
	\xfield{$(f^{-1})'$ (dérivée de la fonction réciproque) ?}
	\xfield{$(f^{-1})' =  \frac{1}{f'(f^{-1})}$}
\end{note}

\begin{note}
	\xfield{Comment trouver le rayon de convergence d'une série entière ?}
	\xfield{On peut le trouver en utilisant les critères de Cauchy et d'Alembert :\begin{itemize}
	\item $r = \left[ \lim\limits_{k \to \infty} (\vert a_k \vert^{\frac{1}{k}})\right]^{-1}$ (Cauchy)
	\item $r=  \lim\limits_{k \to \infty} \big| \frac{a_k}{a_{k+1}}\big|$ (D'Alembert)
	\end{itemize}
	 Si ces limites existent.}
\end{note}

\begin{note}
	\xfield{Soit $f(x) :=  \sum\limits^{\infty}_{k=0}a_k (x-a)^k$ et soit le rayon de convergence $r > 0$. Que peut-t'on dire de
	\begin{itemize}
		\item $f^{(n)}(x)$ ?
		\item $f^{(n)}(a)$ ?
		\item $a_k$ ?
	\end{itemize}}
	\xfield{\begin{itemize}
		\item $f^{(n)}(x) =  \sum\limits^{\infty}_{k=0} a_{k+n} \frac{(k+n)!}{k!} (x-a)^k$
		\item $f^{(n)}(a) = a_n \cdot n!$
		\item $a_k = \frac{1}{k!} f^{(k)}(a)$
	\end{itemize} }
\end{note}

\begin{note}
	\xfield{Donner le théorème de Taylor (développement limité d'ordre $n$)}
	\xfield{Soit la fonction de classe $C^{n+1}(I)$ pour un $n \in \mathbb{N}$ et soit $a,x \in I$ Alors\\
	$f(x) = \sum\limits^{\infty}_{k=0} a_{k+n} a_k (x-a)^k + R_n(a,x)$\\
	avec $a_k = \frac{1}{k!} f^{(k)}(a)$ et avec\\
	$R_n(a,x) = \frac{1}{(n+1)!} f^{(n+1)}(u) (x-a)^{n+1}$\\
	où $u \in ]a,x[$ si $x>a$ et $u \in ]x,a[$ si $x<a$}
\end{note}

\begin{note}
	\xfield{Définir ce qu'est une série de Taylor et une série de MacLaurin.}
	\xfield{Si $f$ est de classe $C^{\infty}(I)$ et si $\lim\limits_{n \to \infty} R_n(a,x)=0$ on obtient à parti du théorème de Taylor (pour x fixe)\\
	$f(x) = \sum\limits^{\infty}_{k=0} \frac{f^{(k)}(a)}{k!}(x-a)^k$ (série de Taylor)\\
	et si $a=0$\\
	$f(x) = \sum\limits^{\infty}_{k=0} \frac{f^{(k)}(0)}{k!}x^k$ (série de MacLaurin)}
\end{note}

\begin{note}
	\xfield{Donner le développement en série entière de
	\begin{enumerate}
		\item $\exp(x)$
		\item $a^x$
		\item $\ln(1-x)$
		\item $\ln(1+x)$
	\end{enumerate}}
	\xfield{\begin{enumerate}
		\item $e^x=\sum\limits_{n=0}^{+{\infty}}{\frac{x^n}{n!}} = 1 + \frac{x}{1!} + \frac{x^2}{2!} + \frac{x^3}{3!} + \frac{x^4}{4!} + ... + \frac{x^n}{n!} + ...$ $\forall x \in \mathbb{R}$
		\item $a^x= e^{x \ln a} =\sum\limits_{n=0}^{+{\infty}}{\frac{(\ln a)^n}{n!}}x^n = 1 + \frac{\ln a}{1!}x + \frac{(\ln a )^2}{2!}x^2 + \frac{(\ln a )^3}{3!}x^3 + \frac{(\ln a )^4}{4!}x^4 + ... + \frac{(\ln a )^n}{n!}x^n + ...$
		\item $\ln (1-x)=-\sum\limits_{n=1}^{+{\infty}}{x^{n}\over{n}} = - \left( x + \frac{x^2}{2} + \frac{x^3}{3} + \frac{x^4}{4} + ...\right)$ $\forall x \in ]-1,1[$
		\item $\ln (1+x)=\sum\limits_{n=1}^{+{\infty}}(-1)^{n-1} \cdot {x^{n}\over{n}} = x - \frac{x^2}{2} + \frac{x^3}{3} - \frac{x^4}{4} + ...$  $\forall x \in ]-1,1[$
	\end{enumerate}}
\end{note}

\begin{note}
	\xfield{Donner le développement en série entière de
	\begin{enumerate}
	\item $\sin(x)$
	\item $\sinh(x)$
	\item $\cos(x)$
	\item $\cosh(x)$
	\end{enumerate} }
	\xfield{\begin{enumerate}
		\item $ \sin x=\sum\limits_{n=0}^{+{\infty}}(-1)^n\,{\frac{x^{2n+1}}{(2\,n+1)!}} = x - \frac{x^3}{3!} + \frac{x^5}{5!} - \frac{x^7}{7!} + ...$ $\forall x \in \mathbb{R}$
		\item $\sinh x = \sum\limits_{n=0}^{+{\infty}}{\frac{x^{2n+1}}{(2\,n+1)!}} = x + \frac{x^3}{3!} +  \frac{x^5}{5!} +  \frac{x^7}{7!} + ...$ $\forall x \in \mathbb{R}$
		\item $\cos x=\sum\limits_{n=0}^{+{\infty}}(-1)^n\,{\frac{x^{2n}}{(2\,n)!}} = 1 - \frac{x^2}{2!} +\frac{x^4}{4!} -\frac{x^6}{6!} +...$ $\forall x \in \mathbb{R}$
		\item $\cosh x = \sum\limits_{n=0}^{+{\infty}}{\frac{x^{2n}}{(2\,n)!}} = 1 +  \frac{x^2}{2!} +  \frac{x^4}{4!} + \frac{x^6}{6!} + ...$ $\forall x \in \mathbb{R}$
	\end{enumerate} }
\end{note}

\begin{note}
	\xfield{Donner le développement en série entière de $(1+x)^\alpha$, $\alpha \in \mathbb{R}$}
	\xfield{$(1+x)^\alpha = \sum\limits_{n=0}^{\alpha}{{\alpha \choose n}\, x^n}$}
\end{note}

\begin{note}
	\xfield{$\int x^n$}
	\xfield{$\frac{1}{n+1}x^{n+1} + C$, $n\neq 1, C\in \mathbb{R}$}
\end{note}

\begin{note}
	\xfield{$\int \frac{1}{x}$}
	\xfield{$\ln(\vert x\vert ) + C$, $x \in \mathbb{R}^*$}
\end{note}

\begin{note}
	\xfield{$\int e^x$}
	\xfield{$e^x + C$}
\end{note}

\begin{note}
	\xfield{$\int \ln(x)$}
	\xfield{$x\ln(x)-x+C$, $x>0$}
\end{note}

\begin{note}
	\xfield{$\int \frac{f'(x)}{f(x)}$}
	\xfield{$\ln(\vert f(x) \vert ) +C$}
\end{note}

\begin{note}
	\xfield{$\int e^{x^2}2x$}
	\xfield{$e^{x^2}+C$}
\end{note}

\begin{note}
	\xfield{$\int \cos(x)$}
	\xfield{$\sin(x) +C$}
\end{note}

\begin{note}
	\xfield{$\int \sin(x)$}
	\xfield{$-\cos(x) +C$}
\end{note}

\begin{note}
	\xfield{$\int \tan(x) = \frac{\sin(x)}{\cos(x)}$}
	\xfield{$-\ln(\vert \cos(x) \vert ) + C$, $x \in D(\tan)$}
\end{note}

\begin{note}
	\xfield{$\int \frac{1}{1+x^2}$}
	\xfield{$\arctan(x) + C$}
\end{note}

\begin{note}
	\xfield{$\int \frac{f'(x)}{1+f(x)^2}$}
	\xfield{$\arctan(f(x)) +C$}
\end{note}


\begin{note}
	\xfield{Donner le théorème de la moyenne pour la fonction $f: \mathbb{R} \to \mathbb{R},\ [a,b] \subset D(f),\ b>a$, $f$ continue sur $[a,b]$.}
	\xfield{Soit $f: \mathbb{R} \to \mathbb{R},\ [a,b] \subset D(f),\ b>a$, $f$ continue sur $[a,b]$. Alors il existe $u \in ]a,b[$. tel que\\
	$\int_a^b f(x) \cdot dx = f(u) (b-a)$   $(=f(u) \int_a^b 1 \cdot dx$\\
	$f(u) = \frac{1}{b-a} \int_a^b f(x) dx$ = valeur moyenne de $f$ sur $[a,b]$}
\end{note}

\begin{note}
	\xfield{Donner le théorème fondamental du calcul intégral.}
	\xfield{Soit $f: \mathbb{R} \to \mathbb{R},\ [a,b] \subset D(f)$, $f$ continue sur $[a,b]$. Alors 
	\begin{itemize}
		\item La fonction $G$ définie par $G(x) = \int_a^x f(t) dt$ est une primitive de $f$ sur $[a,b]$, c.-à-d. $G$ est dérivable sur $]a,b[$ et $G'(x) = f(x)$,c.-à-d.\\
		$\frac{d}{dx}\left(\int_a^x f(t) dt\right) = f(x)$		$\forall x \in ]a,b[$.
		\item Si $F$ est une primitive de $f$ sur $[a,b]$ alors \\
		$\int_a^b f(x) dx = F(b) - F(a)$
	\end{itemize} }
\end{note}

\begin{note}
	\xfield{Donner le théorème d'intégration par changement de variable.}
	\xfield{Soit $f: \mathbb{R} \to \mathbb{R},\ [a,b] \subset D(f)$, $f$ continue sur $[a,b]$.\\
	Soit $\varphi : [\alpha,\beta] \to [a,b]$, $\varphi$ continûment dérivable sur $[\alpha,\beta]$, $\varphi(\alpha) = a$, $\varphi(\beta) = b$\\
	Alors : $\int_a^b f(x) dx = \int_\alpha^\beta f(\varphi(u))\varphi'(u) du$\\
	\emph{Remarque}: Si $\varphi$ est bijectif alors $F(x) = G(\varphi^{-1}(x))$ }
\end{note}

\begin{note}
	\xfield{Donner le théorème d'intégration par partie.}
	\xfield{Soit $f,g : [a,b] \to \mathbb{R}$ continûment dérivable sur $[a,b]$ (=dérivable avec une fonction dérivée qui est continue). Alors :\\
	$\int_a^b f'(x)g(x) dx = \big[f(x)\cdot g(x)\big]_a^b - \int_a^b f(x)g'(x) dx$\\
	Ou en pratique :\\
	$\int_a^b f(x)g(x) dx = \big[ F(x) g(x)\big]_a^b - \int_a^b F(x) g'(x) dx$\\
	avec $f$ continue sur $[a,b]$ et $g$ continûment dérivable sur $[a,b]$. $F$ une primitive de $f$.\\
	et dans le cas d'une intégrale indéfinie :\\
	$\int2 f(x)g(x) dx = F(x)g(x) - \int F(x)g'(x) dx$ }
\end{note}

\begin{note}
	\xfield{Expliquer ce qu'est la méthode de bissection}
	\xfield{On considère deux nombres $a$ et $b$ et une fonction $f$ continue sur l'intervalle $[a, b]$ telle que $f$($a$) et $f(b)$ soient de signes opposés.\\
Supposons que nous voulions résoudre l’équation $f(x) = 0$.\\
Nous savons d'après le théorème des valeurs intermédiaires que $f$ doit avoir au moins un zéro dans l’intervalle $[a, b]$.\\
La méthode de bissection consiste à diviser l’intervalle en deux en calculant $c = \frac{(a+b)}{2}$.\\
Il y a maintenant deux possibilités: ou $f(a)$ et $f(c)$ sont de signes contraires, ou $f(c)$ et $f(b)$ sont de signes contraires.\\
L’algorithme de dichotomie est alors appliqué au sous-intervalle dans lequel le changement de signe se produit, ce qui signifie que l’algorithme de dichotomie est en soi récursif.\\
L’erreur absolue de la méthode de dichotomie est au plus de \\
$\frac{b-a}{2^{n+1}}$}
\end{note}

\begin{note}
	\xfield{Dérivée de :
	\begin{enumerate}
		\item $(f(x)\cdot g(x))'$
		\item $f(g(x))'$
\end{enumerate}	}
	\xfield{\begin{enumerate}
		\item $(f(x)\cdot g(x))' =  f'(x)\cdot g(x) + f(x)\cdot g'(x)$
		\item $f(g(x))' = f'(g(x))\cdot g'(x)$
	\end{enumerate} }
\end{note}

\begin{note}
	\xfield{$\sin(x)'$ ?}
	\xfield{$\cos(x)$}
\end{note}

\begin{note}
	\xfield{$\cos(x)'$ ?}
	\xfield{$-\sin(x)$}
\end{note}

\begin{note}
	\xfield{$\tan(x)'$ ?}
	\xfield{$\frac{1}{\cos^2(x)}$}
\end{note}

\begin{note}
	\xfield{$\arcsin(x)'$ ?}
	\xfield{$\frac{1}{\sqrt{1-x^2}}$}
\end{note}

\begin{note}
	\xfield{$\arccos(x)'$ ?}
	\xfield{$\frac{-1}{\sqrt{1-x^2}}$}
\end{note}


\begin{note}
	\xfield{$\arctan(x)'$ ?}
	\xfield{$\frac{1}{x^2+1}$}
\end{note}

\begin{note}
	\xfield{$(\frac{1}{x})'$ ?}
	\xfield{$-\frac{1}{x^2}$}
\end{note}

\begin{note}
	\xfield{$(a^x)'$ ?}
	\xfield{$a^x \ln(a)$}
\end{note}

\begin{note}
	\xfield{$(\sqrt[x]{a})'$ ?}
	\xfield{$-\frac{\sqrt[x]{a}\ln a}{x^2}$}
\end{note}

\begin{note}
	\xfield{$(\log_a x)'$ ?}
	\xfield{$\frac{1}{x\ln a}$}
\end{note}


\end{document}
